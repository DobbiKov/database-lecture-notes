% Solutions Examen Bases de Données 2021-2022
\documentclass[a4paper,11pt]{article}
\usepackage[utf8]{inputenc}
\usepackage[T1]{fontenc}
\usepackage[french]{babel}
\usepackage{amsmath,amssymb,amsthm}
\usepackage{geometry}
\geometry{margin=1in}

\title{\textbf{Solutions de l'Examen Bases de Données (Session 1, 2021--2022)}}
\date{}

\begin{document}
\maketitle

\section*{A. Requêtes : Algèbre relationnelle et SQL sans sous-requêtes ni agrégats}
\subsection*{1. Marques de bus pouvant transporter plus de 50 voyageurs}
\textbf{Algèbre relationnelle}\[
  \pi_{\text{Marq}}\Bigl(\sigma_{\text{Cap}>50}(\text{Bus})\Bigr)
\]
\textit{Explication :} On sélectionne d'abord les tupples de \texttt{Bus} dont \texttt{Cap>50}, puis on projette l'attribut \texttt{Marq} pour obtenir les marques.

\textbf{SQL}
\begin{verbatim}
SELECT DISTINCT Marq
FROM Bus
WHERE Cap > 50;
\end{verbatim}
\textit{Explication :} La clause \texttt{WHERE} filtre les bus de capacité supérieure à 50, et \texttt{DISTINCT} élimine les doublons.

\subsection*{2. Identifiants des trajets effectués en bus de marque \texttt{RW}}  
\textbf{Algèbre relationnelle}\[
  \pi_{\text{IdT}}\bigl(\sigma_{\text{Marq}='RW'}(\text{Bus}) \;\bowtie\; \text{Traj}\bigr)
\]
\textit{Explication :} On réalise la jointure naturelle entre \texttt{Bus} et \texttt{Traj} pour associer chaque trajet à son bus, on sélectionne les bus de marque \texttt{RW}, puis on projette \texttt{IdT}.

\textbf{SQL}
\begin{verbatim}
SELECT T.IdT
FROM Traj T JOIN Bus B ON T.IdB = B.IdB
WHERE B.Marq = 'RW';
\end{verbatim}
\textit{Explication :} \texttt{JOIN} lie les deux tables ; la clause \texttt{WHERE} filtre sur la marque, et on affiche l'identifiant du trajet.

\subsection*{3. Trajets (IdT) avec bus stationné à Lille et chauffeurs nommés `Driver`}  
\textbf{Algèbre relationnelle}\[
  \pi_{\text{IdT}}\bigl(\sigma_{\text{Ville}='Lille'}(\text{Bus}) \;\bowtie_{B.IdB=Tr.IdB}\; \sigma_{\text{NomC}='Driver'}(\text{Cond}) \;\bowtie_{Tr.IdC=C.IdC}\; \text{Traj}\bigr)
\]
\textit{Explication :} On restreint d'abord \texttt{Bus} à Lille et \texttt{Cond} au nom `Driver`, on effectue les jointures successives avec \texttt{Traj}, puis on projette \texttt{IdT}.

\textbf{SQL}
\begin{verbatim}
SELECT T.IdT
FROM Traj T
JOIN Bus B   ON T.IdB = B.IdB
JOIN Cond C  ON T.IdC = C.IdC
WHERE B.Ville = 'Lille'
  AND C.NomC = 'Driver';
\end{verbatim}

\subsection*{4. Identifiants des bus jamais utilisés sur un trajet}
\textbf{Algèbre relationnelle}\[
  \pi_{\text{IdB}}(\text{Bus})
  \; - \;
  \pi_{\text{IdB}}(\text{Traj})
\]
\textit{Explication :} On prend l'ensemble des bus puis on retire ceux qui apparaissent dans \texttt{Traj}.

\textbf{SQL}
\begin{verbatim}
SELECT IdB
FROM Bus
WHERE IdB NOT IN (SELECT DISTINCT IdB FROM Traj);
\end{verbatim}
\textit{Explication :} La sous-requête liste les bus utilisés, et \texttt{NOT IN} sélectionne les autres.

\subsection*{5. Conducteurs (IdC, NomC) ayant utilisé une marque parmi celles des bus partis de Nice le 14-04-2022}
\textbf{Algèbre relationnelle}
\[
  \pi_{\text{IdC,NomC}}\Bigl(\bigl( \pi_{\text{IdC,IdB}}(\sigma_{\text{VD}='Nice' \land \text{Date}='2022-04-14'}(\text{Traj}))
  \bowtie \text{Traj} \bowtie \text{Bus} \bigr)\Bigr)
\]
\textit{Explication :} On identifie d'abord les trajets partant de Nice ce jour-là, on récupère les marques des bus correspondants, puis on sélectionne les conducteurs ayant conduit ces bus.

\textbf{SQL}
\begin{verbatim}
SELECT DISTINCT C.IdC, C.NomC
FROM Traj T
JOIN Bus B  ON T.IdB = B.IdB
JOIN Cond C ON T.IdC = C.IdC
WHERE T.VD = 'Nice'
  AND T.Date = '2022-04-14'
  AND B.Marq IN (
      SELECT DISTINCT B2.Marq
      FROM Traj T2
      JOIN Bus B2 ON T2.IdB = B2.IdB
      WHERE T2.VD = 'Nice'
        AND T2.Date = '2022-04-14'
  );
\end{verbatim}

\subsection*{6. Villes rejoignables depuis Nice avec un changement à Lyon le même jour}
\textbf{Algèbre relationnelle}\[
  \pi_{\text{Date}, HD1, HA1, VA2, HA2}\Bigl(
    \rho_{T1}(\sigma_{\text{VD}='Nice'}(\text{Traj}))
    \bowtie_{T1.Date=T2.Date \land T1.VA='Lyon' \land T1.HA < T2.HD}
    \rho_{T2}(\sigma_{\text{VD}='Lyon'}(\text{Traj}))
  \Bigr)
\]
\textit{Explication :} On prend deux instances de \texttt{Traj} : \texttt{T1} pour le tronçon Nice→Lyon, \texttt{T2} pour Lyon→destination, on impose même date et correspondance horaire, puis on projette les attributs demandés.

\textbf{SQL}
\begin{verbatim}
SELECT T1.Date,
       T1.HD    AS HeureDepartNice,
       T1.HA    AS HeureArriveeLyon,
       T2.VA    AS VilleFinale,
       T2.HA    AS HeureArriveeFinale
FROM Traj T1
JOIN Traj T2 ON T1.Date = T2.Date
             AND T1.VA = 'Lyon'
             AND T1.HA < T2.HD
WHERE T1.VD = 'Nice'
  AND T2.VD = 'Lyon';
\end{verbatim}

\subsection*{7. Conducteurs ayant fait au moins deux trajets entre Lille et Nice}
\textbf{Algèbre relationnelle}\[
  \pi_{\text{IdC}}\Bigl(\sigma_{\text{VD}='Lille' \land \text{VA}='Nice'}(\text{Traj})
   \;
   \bowtie_{IdC}
   \sigma_{\text{VD}='Nice' \land \text{VA}='Lille'}(\text{Traj})\Bigr)
\]
\textit{Explication :} On réalise la jointure sur \texttt{IdC} entre trajets Lille→Nice et Nice→Lille, ce qui garantit au moins deux trajets.

\textbf{SQL}
\begin{verbatim}
SELECT C.IdC
FROM (
  SELECT IdC
  FROM Traj
  WHERE VD = 'Lille' AND VA = 'Nice'
) AS A
JOIN (
  SELECT IdC
  FROM Traj
  WHERE VD = 'Nice' AND VA = 'Lille'
) AS B ON A.IdC = B.IdC;
\end{verbatim}

\section*{B. Algèbre relationnelle uniquement}
\subsection*{8. Villes où tous les conducteurs ont démarré au moins un trajet}
\textbf{Algèbre relationnelle}\[
  \pi_{\text{VD}}(\text{Traj})
  \; -
  \pi_{\text{VD}}\bigl(\pi_{\text{VD}}(\text{Traj}) \times (\pi_{\text{IdC}}(\text{Cond})
    \; -
  \pi_{\text{IdC}}(\text{Traj}))\bigr)
\]
\textit{Explication :} On énumère d'abord toutes les villes de départ, puis on retire celles pour lesquelles il existe au moins un conducteur n'ayant pas de trajet débutant dans cette ville.

\section*{C. SQL avec agrégats et sous-requêtes}
\subsection*{9. Nombre de trajets au départ de Nice le 14-04-2022}
\begin{verbatim}
SELECT COUNT(*) AS NbTrajets
FROM Traj
WHERE VD = 'Nice'
  AND Date = '2022-04-14';
\end{verbatim}
\textit{Explication :} \texttt{COUNT(*)} compte les lignes satisfaisant la condition.

\subsection*{10. Nombre de trajets par marque (supérieur à 20)}
\begin{verbatim}
SELECT B.Marq, COUNT(*) AS NbTrajets
FROM Traj T
JOIN Bus B ON T.IdB = B.IdB
GROUP BY B.Marq
HAVING COUNT(*) > 20;
\end{verbatim}
\textit{Explication :} \texttt{GROUP BY} regroupe par marque, et \texttt{HAVING} filtre les groupes de taille > 20.

\section*{Normalisation}
\textbf{Schéma :}\newline
Agence(\underline{IdCl}, NomCl, \underline{IdApp}, AdrApp, DDLoc, DFLoc, Montant, IdProp, NomProp)\\
Dépendances fonctionnelles :\\
\begin{tabular}{ll}
1. & IdCl $\to$ NomCl \\
2. & IdProp $\to$ NomProp \\
3. & NomProp $\to$ IdProp \\
4. & IdCl,IdApp $\to$ DDLoc,DFLoc \\
5. & IdApp $\to$ AdrApp, Montant, IdProp, NomProp
\end{tabular}

\subsection*{Q1. Instance illustrant anomalies}
\begin{center}
\begin{tabular}{|c|c|c|c|c|c|c|c|c|}
\hline
IdCl & NomCl & IdApp & AdrApp & DDLoc & DFLoc & Montant & IdProp & NomProp \\
\hline
1 & Alice   & A1 & 1 rue X & 2022-01-01 & 2022-02-01 & 500 & P1 & Dupont \\
1 & Alice   & A2 & 2 rue Y & 2022-03-01 & 2022-04-01 & 300 & P2 & Martin \\
2 & Bob     & A1 & 1 rue X & 2022-05-01 & 2022-06-01 & 500 & P1 & Dupont \\
3 & Charlie & A3 & 3 rue Z & 2022-02-15 & 2022-03-15 & 400 & P3 & Durand \\
3 & Charlie & A2 & 2 rue Y & 2022-07-01 & 2022-08-01 & 300 & P2 & Martin \\
\hline
\end{tabular}
\end{center}

\textbf{Type 1 :} \emph{Insertion}\\
\textit{Problème :} Impossible d'ajouter un nouvel appartement A4 sans louer à un client existant (puisqu'AdrApp dépend de IdApp dans la même table).\\[1em]

\textbf{Type 2 :} \emph{Suppression}\\
\textit{Problème :} Si on supprime la dernière location de A3, on perd l'adresse de A3 et les informations sur le propriétaire P3.\\[1em]

\textbf{Type 3 :} \emph{Mise à jour}\\
\textit{Problème :} Si le propriétaire P2 change de nom (ex. `Martin` → `Petit`), il faut mettre à jour plusieurs lignes (pour A2, A2), d'où risque d'incohérence.

\subsection*{Q2. Clés du schéma}
Un déterminant minimal qui couvre tous les attributs : \{IdCl, IdApp\}.\\
\textit{Justification :} 
\begin{itemize}
  \item De IdCl,IdApp découlent DDLoc,DFLoc (DF 4).
  \item De IdCl découle NomCl (DF 1).
  \item De IdApp découlent AdrApp, Montant, IdProp, NomProp (DF 5), et de IdProp ⇄ NomProp (DF 2 et 3).\\
\end{itemize}
Aucune sous-partie propre n'est clé, donc c'est la clé candidate.

\subsection*{Q3. 3ème forme normale}
Non. Par exemple, IdCl $\to$ NomCl viole la 3FN car IdCl n'est pas super-clé et NomCl n'est pas partie d'une clé.

\subsection*{Q4. Forme normale de Boyce–Codd}
Non plus, car pour IdProp $\to$ NomProp, IdProp n'est pas super-clé du schéma complet.

\subsection*{Q5. Décomposition en FNBC}
On applique l'algorithme de décomposition :\\
1. FD IdCl $\to$ NomCl donne R1(IdCl,NomCl).\\
2. FD IdApp $\to$ AdrApp,Montant,IdProp,NomProp donne R2(IdApp,AdrApp,Montant,IdProp,NomProp).\\
3. FD IdCl,IdApp $\to$ DDLoc,DFLoc donne R3(IdCl,IdApp,DDLoc,DFLoc).\\
4. FD IdProp $\to$ NomProp donne R4(IdProp,NomProp) (mais NomProp déjà dans R2).\\
On conserve l'information et chaque schéma est maintenant en BCNF.

\end{document}
