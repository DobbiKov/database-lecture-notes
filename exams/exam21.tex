% Préambule
\documentclass[a4paper,12pt]{article}
\usepackage[french]{babel}
\usepackage[utf8]{inputenc}
\usepackage[T1]{fontenc}
\usepackage{amsmath,amssymb}
\usepackage{geometry}
\geometry{margin=2.5cm}
\title{Solutions de l'examen \textit{Bases de données} 2020--2021}
\author{}
\date{}

\begin{document}
\maketitle

%-----------------------
\section*{Exercice A : Algèbre relationnelle et SQL sans imbrication}

\subsection*{1. Nom des équipements situés au repère GPS \texttt{345\_78S}}
\paragraph{Algèbre relationnelle}
\[
  \pi_{Nom}\bigl(\sigma_{GPS='345\_78S'}(Equip)\bigr)
\]
\paragraph{SQL}
\begin{verbatim}
SELECT Nom
FROM Equip
WHERE GPS = '345_78S';
\end{verbatim}
\paragraph{Explication}
On sélectionne d'abord les tuples de \texttt{Equip} dont \texttt{GPS='345\_78S'} (opérateur \(\sigma\)), puis on projette la colonne \texttt{Nom} (opérateur \(\pi\)).

\subsection*{2. Identifiants des épreuves de ski}
\paragraph{Algèbre relationnelle}
\[
  \pi_{IdE}\Bigl(Epreuve \\
  \quad\bowtie_{Epreuve.IdS = Sport.IdS}  \sigma_{Nom='ski'}(Sport)\Bigr)
\]
\paragraph{SQL}
\begin{verbatim}
SELECT E.IdE
FROM Epreuve AS E
JOIN Sport AS S
  ON E.IdS = S.IdS
WHERE S.Nom = 'ski';
\end{verbatim}
\paragraph{Explication}
On filtre les sports de nom ``ski'', on les joint à \texttt{Epreuve} selon \texttt{IdS}, puis on projette \texttt{IdE}.

\subsection*{3. Identifiants des épreuves individuelles au GPS \texttt{345\_78S}}
\paragraph{Algèbre relationnelle}
\[
\begin{array}{l}
  \pi_{IdE}\Bigl(\sigma_{TP='individuelle'}(Sport) \\
  \quad\bowtie_{Sport.IdS=Epreuve.IdS}
  Epreuve \\
  \quad\bowtie_{Epreuve.IdEq=Equip.IdEq}  \sigma_{GPS='345\_78S'}(Equip)\Bigr)
\end{array}
\]
\paragraph{SQL}
\begin{verbatim}
SELECT E.IdE
FROM Sport AS S
JOIN Epreuve AS E
  ON S.IdS = E.IdS
JOIN Equip AS Q
  ON E.IdEq = Q.IdEq
WHERE S.TP = 'individuelle'
  AND Q.GPS = '345_78S';
\end{verbatim}
\paragraph{Explication}
On applique deux sélections successives (type ``individuelle'', puis GPS) via deux jointures, puis on projette \texttt{IdE}.

\subsection*{4. Sports (nom) sans aucune épreuve}
\paragraph{Algèbre relationnelle}
\[
  \pi_{Nom}(Sport) - \pi_{Nom}\bigl(Sport \bowtie Epreuve\bigr)
\]
\paragraph{SQL}
\begin{verbatim}
SELECT Nom
FROM Sport
WHERE IdS NOT IN (
  SELECT DISTINCT S.IdS
  FROM Sport AS S
  JOIN Epreuve AS E
    ON S.IdS = E.IdS
);
\end{verbatim}
\paragraph{Explication}
On prend la différence entre l'ensemble des sports et ceux apparaissant dans \texttt{Epreuve}.

\subsection*{5. Sports (nom, TP) ayant utilisé l'équipement de l'épreuve de ski individuelle du \num{14-02-2014}}
\paragraph{Algèbre relationnelle}
\[
  \pi_{Nom,TP}\Bigl(
    Sport \bowtie \\
    \bigl(\pi_{IdEq}(\sigma_{Nom='ski' \wedge TP='individuelle' \wedge Date='2014-02-14'}(Epreuve \bowtie Sport))\bigr)
  \Bigr)
\]
\paragraph{SQL}
\begin{verbatim}
SELECT DISTINCT S.Nom, S.TP
FROM Sport AS S
JOIN Epreuve AS E
  ON S.IdS = E.IdS
WHERE E.IdEq = (
  SELECT IdEq
  FROM Epreuve AS X
  JOIN Sport AS Y
    ON X.IdS = Y.IdS
  WHERE Y.Nom = 'ski'
    AND Y.TP = 'individuelle'
    AND X.Date = '2014-02-14'
)
;
\end{verbatim}
\paragraph{Explication}
On isole d'abord l'\texttt{IdEq} de l'épreuve ciblée par une sous-requête, puis on sélectionne tous les sports l'ayant utilisé.

\subsection*{6. Équipements jamais utilisés (noms)}
\paragraph{Algèbre relationnelle}
\[
  \pi_{Nom}(Equip) - \pi_{Nom}(Equip \bowtie Epreuve)
\]
\paragraph{SQL}
\begin{verbatim}
SELECT Nom
FROM Equip
WHERE IdEq NOT IN (
  SELECT DISTINCT IdEq
  FROM Epreuve
);
\end{verbatim}
\paragraph{Explication}
On liste les équipements dont l'identifiant n'apparaît pas dans la relation \texttt{Epreuve}.

\subsection*{7. Identifiants des équipements utilisés au moins deux fois}
\paragraph{Algèbre relationnelle}

La condition « au moins deux fois » s'exprime par auto-mutlijoin :
\[
  \pi_{E1.IdEq}\bigl(\sigma_{E1.IdEq = E2.IdEq \wedge E1.IdE \neq E2.IdE}(\rho_{E1}(Epreuve) \times \rho_{E2}(Epreuve))\bigr)
\]
\paragraph{SQL}
\begin{verbatim}
SELECT IdEq
FROM Epreuve
GROUP BY IdEq
HAVING COUNT(*) >= 2;
\end{verbatim}
\paragraph{Explication}
On regroupe par \texttt{IdEq} et on filtre ceux dont le nombre d'occurrences est au moins \num{2}.

%-----------------------
\section*{Exercice B : Algèbre relationnelle (division)}

\subsection*{8. Lieux (GPS) où \emph{tous} les sports ont été programmés}
Soit
\begin{itemize}
  \item $R_1 = \pi_{IdS,IdEq}(Epreuve)$
  \item $R_2 = \pi_{IdS}(Sport)$
  \item $R_3 = \pi_{GPS,IdEq}(Equip)$
\end{itemize}
On veut
\[
  \pi_{GPS} \Bigl(R_3 \bowtie (R_1 \div R_2)\Bigr)
\]
où $R_1 \div R_2$ donne les $IdEq$ associés à \emph{tous} les $IdS$. Enfin on retrouve les GPS correspondants.

%-----------------------
\section*{Exercice C : SQL avec agrégats et sous-requêtes}

\subsection*{9. Nombre d'équipements au repère GPS \texttt{345\_78S}}
\begin{verbatim}
SELECT COUNT(*) AS nb_equipements
FROM Equip
WHERE GPS = '345_78S';
\end{verbatim}

\subsection*{10. Identifiants des équipements utilisés au moins 10 fois}
\begin{verbatim}
SELECT IdEq
FROM Epreuve
GROUP BY IdEq
HAVING COUNT(*) >= 10;
\end{verbatim}

\subsection*{11. Durée moyenne des épreuves par sport}
\begin{verbatim}
SELECT S.IdS, S.Nom, AVG(E.Duree) AS duree_moyenne
FROM Sport AS S
JOIN Epreuve AS E
  ON S.IdS = E.IdS
GROUP BY S.IdS, S.Nom;
\end{verbatim}

\subsection*{12. Lieux (GPS) où \emph{tous} les sports ont été programmés (SQL)}
\begin{verbatim}
SELECT Q.GPS
FROM Equip AS Q
WHERE NOT EXISTS (
  SELECT *
  FROM Sport AS S
  WHERE NOT EXISTS (
    SELECT *
    FROM Epreuve AS E
    WHERE E.IdS = S.IdS
      AND E.IdEq = Q.IdEq
  )
);
\end{verbatim}

%-----------------------
\section*{Anomalies de mise à jour}
On considère la relation \texttt{Employe}(Emp\_Id, Nom, Salle, Extension) dont l'instance est celle de la Figure 2.

\subsection*{1. Trois types d'anomalies}
\begin{itemize}
  \item \textbf{Anomalie de mise à jour (update anomaly)} : pour modifier l'extension de la salle \texttt{A7}, il faut mettre à jour deux tuples (Dupont, Petit) : risque d'incohérence.
  \item \textbf{Anomalie de suppression (delete anomaly)} : si on supprime la ligne de \texttt{Grand}, on perd l'information qu'il existe une extension \texttt{4501} pour la salle \texttt{B4} si aucun autre employé n'est dans \texttt{B4}.
  \item \textbf{Anomalie d'insertion (insert anomaly)} : on ne peut pas enregistrer une nouvelle salle \texttt{C1} avec extension \texttt{3200} avant qu'un employé ne l'occupe : la relation lie employés et salles.
\end{itemize}

\subsection*{2. Élimination des anomalies}
On décompose en deux tables :
\[
  \begin{aligned}
    Employe'(Emp\_Id, Nom, Salle),\\
    Salle(Salle, Extension)
  \end{aligned}
\]
Ainsi l'extension dépend uniquement de la salle.

\subsection*{3. Nécessité de la normalisation}
La normalisation est justifiée si l'on souhaite garantir l'
intégrité et éviter les anomalies à long terme. Toutefois, pour de très petites relations
ou en cas de fortes contraintes de performance,
on peut tolérer un certain dénormalisation.

%-----------------------
\section*{Formes normales}

\subsection*{Cas 1 : R(ABCD), $F=\{C \to D,\,C \to A,\,B \to C\}$}

\subsubsection*{1. $B^+$}
On a $B\to C$, $C\to A$, $C\to D$, donc
\[
  B^+ = \{B,C,A,D\}.
\]

\subsubsection*{2. $F^+$ (fermée de $F$)}
Outre les dépendances initiales, on déduit :
\[
  B\to A,
  \quad B\to D,
  \quad (...)\quad
\]
(on liste l'ensemble des DF impliquées par transitivité)

\subsubsection*{3. Clés}
Puisque $B^+$ contient tous les attributs, $B$ est clé. Aucune autre singleton ne l'est.

\subsubsection*{4. 3FN ?}
La DF $C \to A$ viole la 3FN car $C$ n'est pas super-clé et $A$ n'est pas partie d'une clé. Donc \emph{non}.

\subsubsection*{5. Instance redondante}
Par exemple :
\begin{center}
\begin{tabular}{|c|c|c|c|}\hline
B & C & A & D\\\hline
b1 & c1 & a1 & d1\\
b1 & c1 & a1 & d2\\\hline
\end{tabular}
\end{center}
On répète $a1$ à tort.

\subsection*{Cas 2 : R(ABCDE), $F=\{A \to B,\,AB \to D,\,C \to E,\,D \to C,\,AC \to E\}$}

\subsubsection*{1. Clés}
On calcule $A^+$ : $A\to B$, puis $AB\to D$, $D\to C$, $C\to E$. Ainsi $A^+=\{A,B,D,C,E\}$, donc $A$ est clé.

\subsubsection*{2. FNBC ?}
La DF $D\to C$ viole la BCNF (l'antécédent D n'est pas super-clé). Donc \emph{non}.

\subsubsection*{3. Décomposition BCNF}
On choisit la DF violant BCNF, $D\to C$, et on obtient :
\[
  R_1(D,C),
  \quad R_2(A,B,D,E)
\]
Puis dans $R_2$, on a $AB\to D$ qui viole BCNF, on décompose encore :
\[
  R_{21}(A,B,D),
  \quad R_{22}(A,B,E)
\]
Arbre de décomposition :
\[
  R(ABCDE)\;
  \xrightarrow{D\to C}\;
  \bigl(R_1(DC),\;R_2(ABDE)\bigr)
  \xrightarrow{AB\to D}\;
  \bigl(R_{21}(ABD),\;R_{22}(ABE),\;R_1(DC)\bigr)
\]

\end{document}
